\documentclass[fin1, tisk]{fmfdelo}
% \documentclass[fin1, tisk]{fmfdelo}
% Če pobrišete možnost tisk, bodo povezave obarvane,
% na začetku pa ne bo praznih strani po naslovu, …

%%%%%%%%%%%%%%%%%%%%%%%%%%%%%%%%%%%%%%%%%%%%%%%%%%%%%%%%%%%%%%%%%%%%%%%%%%%%%%%
% METAPODATKI
%%%%%%%%%%%%%%%%%%%%%%%%%%%%%%%%%%%%%%%%%%%%%%%%%%%%%%%%%%%%%%%%%%%%%%%%%%%%%%%

% - vaše ime
\avtor{Luka Lavš, Tinka Napret-Kaučič}

% - naslov dela v slovenščini
\naslov{Komplementarni drugi Zagrebški indeks} 

% - naslov dela v angleščini
\title{The Complementary Second Zagreb Index}

% - ime mentorja/mentorice s polnim nazivom:
%   - doc.~dr.~Ime Priimek
%   - izr.~prof.~dr.~Ime Priimek
%   - prof.~dr.~Ime Priimek
%   za druge variante uporabite ustrezne ukaze
\mentor{dr. Riste Škrekovski}
\somentor{dr. Timotej Hrga}

\letnica{december, 2025} 


% - povzetek v slovenščini
%   V povzetku na kratko opišite vsebinske rezultate dela. Sem ne sodi razlaga
%   organizacije dela, torej v katerem razdelku je kaj, pač pa le opis vsebine.
\povzetek{V povzetku na kratko opišemo vsebinske rezultate dela. Sem ne sodi
razlaga organizacije dela -- v katerem poglavju/razdelku je kaj, pač pa le opis
vsebine.}

% - povzetek v angleščini
\abstract{Prevod slovenskega povzetka v angleščino.}

% - klasifikacijske oznake, ločene z vejicami
%   Oznake, ki opisujejo področje dela, so dostopne na strani https://www.ams.org/msc/
\klasifikacija{74B05, 65N99}

% - angleško-slovenski slovar strokovnih izrazov
\slovar{}

% - ime datoteke z viri (vključno s končnico .bib), če uporabljate BibTeX
\literatura{literatura.bib}

%%%%%%%%%%%%%%%%%%%%%%%%%%%%%%%%%%%%%%%%%%%%%%%%%%%%%%%%%%%%%%%%%%%%%%%%%%%%%%%
% DODATNE DEFINICIJE
%%%%%%%%%%%%%%%%%%%%%%%%%%%%%%%%%%%%%%%%%%%%%%%%%%%%%%%%%%%%%%%%%%%%%%%%%%%%%%%

% naložite dodatne pakete, ki jih potrebujete
\usepackage{algpseudocode}  % za psevdokodo
\usepackage{algorithm}      % za algoritme
\usepackage{graphicx}       % za slike
\usepackage{subcaption}
\floatname{algorithm}{Algoritem}
\renewcommand{\listalgorithmname}{Kazalo algoritmov}

% deklarirajte vse matematične operatorje, da jih bo LaTeX pravilno stavil
% \DeclareMathOperator{\conv}{conv}
% na razpolago so naslednja matematična okolja, ki jih kličemo s parom
% \begin{imeokolja}[morebitni komentar v oklepaju] ... \end{imeokolja}
%
% definicija, opomba, primer, zgled, lema, trditev, izrek, posledica, dokaz

% za številske množice uporabite naslednje simbole
\newcommand{\R}{\mathbb R}
\newcommand{\N}{\mathbb N}
\newcommand{\Z}{\mathbb Z}
\newcommand{\KC}{\overline{K}}
\newcommand{\cM}{\operatorname{cM}_2}
\newcommand{\argmax}{\mathop{\mathrm{arg\,max}}\limits}
\newcommand{\CC}{\mathcal{C}_{n, k}}
% Lahko se zgodi, da je ukaz \C definiral že paket hyperref,
% zato dobite napako: Command \C already defined.
% V tem primeru namesto ukaza \newcommand uporabite \renewcommand
\newcommand{\C}{\mathbb C}
\newcommand{\Q}{\mathbb Q}

%%%%%%%%%%%%%%%%%%%%%%%%%%%%%%%%%%%%%%%%%%%%%%%%%%%%%%%%%%%%%%%%%%%%%%%%%%%%%%%
% ZAČETEK VSEBINE
%%%%%%%%%%%%%%%%%%%%%%%%%%%%%%%%%%%%%%%%%%%%%%%%%%%%%%%%%%%%%%%%%%%%%%%%%%%%%%%

\begin{document}

%%%%%%%%%%%% Lavš

\section{Drugi komplementarni Zagrebški indeks}
\begin{definicija}
Naj bo $G$ graf ter $E(G)$ množica njegovih povezav. Z $d_G(u)$ 
označimo stopnjo vozlišča $u \in E(G)$. Tedaj definirajmo drugi 
komplementarni Zagrebški indeks (DKZI) kot:
\begin{equation}
  \cM(G) := \sum_{uv \in E(G)} |\,d_G(u)^2 - d_G(v)^2\,|
\end{equation} 
\end{definicija}

V tem delu se bomo osredotočili na problem, pri katerih povezanih grafih 
DKZI doseže ekstremne vrednosti. Za začetek 
navedimo trditev, ki jo je
dokazal \textcite{proof2025}, leta 2025.

\begin{trditev}\label{tr:domneva}
Naj bo $\mathcal{G}_n$ množica povezanih grafov reda $n$, 
$K_m$ popoln graf, torej $(m-1)$-regularen, in 
$\overline{K}_m$ njegov komplement, torej nepovezan graf z $m$ vozlišči.

Tedaj za vsak $n \geq 5$ velja, da če $G^* \in \mathcal{G}_n$ doseže 
maksimalno vrednost drugega komplementarnega Zagrebškega indeksa 
$\cM(G)$, potem obstaja $k < \lceil n/2 \rceil$, tak da je $G^*$ 
izomorfen grafu $K_k + \overline{K}_{n-k}$. Oziroma krajše:
\begin{equation}
 \forall\, n \geq 5:\; G^* \in \argmax_{G \in \mathcal{G}_n} \cM(G)
  \implies \exists \;k < \lceil \frac{n}{2} \rceil:\  
    G^* \cong K_k + \KC_{n-k}.
\end{equation}
Tu je operacija $+$ definirana kot \emph{join} grafov, torej za grafa $G$ in $H$ je
$G + H$ graf, ki ga dobimo tako, da vzamemo disjunktna grafa $G$ in $H$ 
ter dodamo vse povezave med vsakim vozliščem grafa $G$ in vsakim vozliščem grafa $H$.
\end{trditev}

%%%%%%%%%%%%%%%%%%% Lavš

\section[k kot funkcija parametra n]{$k$ kot funkcija parametra $n$}
Trditev~\ref{tr:domneva} nam pove, da so grafi, ki za 
dano število vozlišč $n$ dosežejo 
maksimalno vrednost drugega komplementarnega Zagrebškega indeksa, oblike 
$K_k + \overline{K}_{n-k}$. Glede parametra $k$ pa zgolj omeji območje iskanja 
na $[0,  \lceil \frac{n}{2} \rceil]$. S tem se ne zadovoljimo prehitro ter 
analitično poiščimo funkcijo $\Gamma$, ki bo za dani parameter $n$ vrnila ustrezno 
število $k$.

Najprej predstavimo DKZI grafa iz trditve~\ref{tr:domneva} 
kot funkcijo parametrov $n$ in $k$. Nato pa dano funkcijo maksimizirajmo, kjer se seveda 
omejimo na naravna števila.

\begin{trditev}\label{tr:fnk}
Naj bo $n \geq k$, tedaj je DKZI 
grafa $G = K_k + \KC_{n-k}$ enak $\cM(G) = k (n-k) ((n-1)^2 - k^2)$.
\end{trditev}

\begin{dokaz}
Naj bo  $G = K_k + \KC_{n-k}$. Graf $G$ je očitno sestavljen iz dveh gruč,
zato njegove povezave razdelimo na notranje povezave 
podgrafa $K_k$, povezave med $K_k$ in 
$\KC_{n-k}$ ter na notranje povezave podgrafa $\KC_{n-k}$.
\begin{align*}
  \cM(G) &= \sum_{uv \in E(G)} |\,d_G(u)^2 - d_G(v)^2\,| \\
  &= \sum_{uv \in E(K_k)} |\,((k-1) + (n-k))^2 - ((k-1) + (n-k))^2\,| + \\
    &\sum_{u \in V(K_k)\, \land \,v \in V(\KC_{n-k})}|\,((k-1)+(n-k))^2 - k^2\,| + \\
    &\sum_{uv \in E(\KC_{n-k})} |\,k^2 - k^2\,| \\
  &= |V(K_k)| |V(\KC_{n-k})| ((n-1)^2 - k^2) \\
  &= k (n-k) ((n-1)^2 - k^2)
\end{align*} 
\end{dokaz}

Sedaj funkcijo imamo, označimo jo z $f(n, k) = k (n-k) ((n-1)^2 - k^2)$.
Najprej poiščimo njen maksimum (pri fiksnem n) na množici realnih števil. 
Pri tem si bomo pomagali s Cardanovo formulo~\cite{cardano} ter z analizo 
odvodov.

\begin{izrek}[Cardanova formula]\label{iz:cardano}
Naj bo $ax^3 + bx^2 + cx + d = 0$ splošna kubična enačba z $a \neq 0$ ter 
naj bodo $\Delta_0 = b^2 - 3ac$, $\Delta_1 = 2b^2 - 9abc + 27a^2d$ in 
$C = \sqrt[3]{\frac{\Delta_1 + \sqrt{\Delta_1^2 - 4 \Delta_0^3}}{2}}$. 
Potem so rešitve dane kubične enačbe enake:
\begin{equation}
  x_j = -\frac{1}{3a}\left(b + \xi^j C + \frac{\Delta_0}{\xi^j C}\right),
\end{equation}
za $\xi = e^{2\pi i/3}$, ter $j \in \{0, 1, 2\}$.
\end{izrek}

\begin{trditev}\label{tr:glavna_k}
Za poljuben $n \in \R$ funkcija $f(n, k) = k (n-k) ((n-1)^2 - k^2)$ 
doseže maksimum pri 

\begin{align*}
k = \frac{1}{12} \Big(3 n  -
\sqrt[3]{3} e^{\frac{4 i \pi}{3}} \sqrt[3]{A} +
\frac{3^{2/3} B}{e^{\frac{4 i \pi}{3}} \sqrt[3]{A}}
 \Big),
\end{align*}
kjer sta $A$ in $B$ definirana kot:
\begin{align*}
A &= 9 (n-2) n (3 n-2) \\
  &\quad + 4 \sqrt{3}\, 
      \sqrt{-(n-1)^2 \bigl(n (n (2 n (34 n-73)+201)-128)+32\bigr)}\ ;\\
B &= (16-11 n)n - 8.
\end{align*}

\noindent Velja tudi, da je $f$ konkavna na 
intervalu 
\[\left[\frac{n}{4} - \frac{\sqrt{11n^2 - 16n +8}}{4\sqrt{3}}\,,
\frac{n}{4} + \frac{\sqrt{11n^2 - 16n +8}}{4\sqrt{3}}\right]
\]
in konveksna drugod.
\end{trditev}

\begin{dokaz}
  V dokazu si bomo pomagali s parcialnima odvodoma funkcije $f$:
\begin{gather}
\frac{\partial f}{\partial k} = 4k^3 - 3nk^2 - 2(n-1)^2k + n(n-1)^2\label{al:f1}\\
\frac{\partial^2 f}{\partial k^2} = 12k^2 - 6nk - 2(n-1)^2\label{al:f2}
\end{gather}

Kvadratna neenačba $\frac{\partial^2 f}{\partial k^2} \leq 0$, glede na parameter $k$,
nam da rešitev 
\begin{equation}
k \in \left[\frac{n}{4} - \frac{\sqrt{11n^2 - 16n +8}}{4\sqrt{3}}\,,
\frac{n}{4} + \frac{\sqrt{11n^2 - 16n +8}}{4\sqrt{3}}\right],\label{eq:interval_konkavnosti}
\end{equation}
$f$ je torej tu res konkavna.

Pokažimo, da so vse rešitve enačbe  $\frac{\partial f}{\partial k} = 0$
realne. To velja, saj sta si vrednosti ekstremov funkcije $\frac{\partial f}{\partial k}$ 
(parametra $k$) nasprotno predznačeni. Velja namreč:
\begin{align*}
\frac{\partial f}{\partial k}(n, \frac{n}{4} - \frac{\sqrt{11n^2 - 16n +8}}{4\sqrt{3}})\cdot 
\frac{\partial f}{\partial k}(n, \frac{n}{4} + \frac{\sqrt{11n^2 - 16n +8}}{4\sqrt{3}}) \\
= -\frac{17}{27} n^6 + \frac{47}{18} n^5 - \frac{187}{36} n^4 + \frac{169}{27} n^3 - \frac{163}{36} n^2 + \frac{16}{9} n - \frac{8}{27}
< 0
\end{align*}

Sedaj pa enačbo $\frac{\partial f}{\partial k} = 0$ razrešimo s pomočjo Cardanove 
formule~\ref{iz:cardano}. Po nekaj izračunih vidimo, 
da je $j$-ta rešitev enačbe enaka:
\begin{align*}
x_j = \frac{1}{12} \Big( \underbrace{3 n}_{\text{prvi člen}}  -
\underbrace{\sqrt[3]{3} e^{\frac{2 i \pi j}{3}} \sqrt[3]{A}}_{\text{drugi člen}} +
\underbrace{\frac{3^{2/3} e^{-\frac{2 i \pi j}{3}} B}{\sqrt[3]{A}}}_{\text{tretji člen}} 
 \Big),
\end{align*}
kjer sta $A$ in $B$ enaka kot v trditvi~\ref{tr:glavna_k}.
S tem zaključimo dokaz.

\end{dokaz}

Z nekaj dodatnimi izračuni lahko še pokažemo, da je funkcija $f$, za vsak 
$n \geq 5$, konkavna tudi na intervalu 
$[0, \lceil\frac{n}{2}\rceil]$. To je res, saj 
\[
\forall n \in \R:\ \frac{n}{4} - \frac{\sqrt{11n^2 - 16n +8}}{4\sqrt{3}} \leq 0.
\]
in tudi:
\[
\forall n \geq 5:\ \frac{n}{4} + \frac{\sqrt{11n^2 - 16n +8}}{4\sqrt{3}} > \lceil\frac{n}{2}\rceil.
\]
Če nam uspe pokazati še, da rešitev $x_j$, ki je po velikosti druga, leži 
na intervalu $[0, \lceil\frac{n}{2}\rceil]$; potem lahko sklepamo, da je 
$\argmax_{1 \leq k \leq \lceil n/2\rceil} f$ v neposredni bližini rešitve $x_j$.

    \begin{figure}[h]
    \centering
    \includegraphics[width=0.6\textwidth]{figures/function_K.pdf}
\caption{
$k^* := \displaystyle \argmax_{x \in [0, \lceil n/2 \rceil]} f(n, x)$ in 
$\Gamma(n) := \displaystyle \argmax_{\substack{x \in [0, \lceil n/2 \rceil] \land x \in \mathbb{N}}} f(n, x)$
%
}
    \label{fig:function_K}
    \end{figure}

In res empirična in asimptotska analiza pokaže, 
da je $x_0 \leq x_2 \leq x_1$ za vsak $n$ ter, 
da $0 < x_2 < \lceil\frac{n}{2}\rceil$ za vsak $n \geq 5$.

Od tu sledi naslednji izrek, katerega smo ravnokar tudi dokazali. Funkcijo $\Gamma$ smo namreč 
našli.

%%%%%%%

\begin{izrek}[$k$ kot funkcija $n$]\label{iz:funkcija_k}
Naj bo $\mathcal{G}_n$ množica povezanih grafov reda $n$. Naj bo 
$\cM : \mathcal{G}_n \rightarrow \R$ funkcija, ki grafu priredi DKZI. 
Naj bo $\Gamma : \N \rightarrow \N$
"funkcija"~ s predpisom:
\begin{equation}
\Gamma(n) = \argmax_{x \in \{ \lceil h(n) \rceil, \lfloor h(n) \rfloor\}} \cM(x),
\end{equation}
kjer so $h$, $A$ ter $B$ podani kot:
\begin{align*}
h(n) = \frac{1}{12} \Big(3 n  -
\sqrt[3]{3} e^{\frac{4 i \pi}{3}} \sqrt[3]{A} +
\frac{3^{2/3} B}{e^{\frac{4 i \pi}{3}} \sqrt[3]{A}}
 \Big);
\end{align*}
\begin{align*}
A &= 9 (n-2) n (3 n-2) \\
  &\quad + 4 \sqrt{3}\, 
      \sqrt{-(n-1)^2 \bigl(n (n (2 n (34 n-73)+201)-128)+32\bigr)}\ ;\\
B &= (16-11 n)n - 8
\end{align*}

Tedaj za vsak $n \geq 5$ velja:
\begin{equation}
  G^* \in \argmax_{G\in\mathcal{G}_n} \cM(G) 
  \implies \exists k \in \Gamma(n):\ G^* \cong K_{k} + \KC_{n-k}
\end{equation}
\end{izrek}

\begin{opomba}
"Funkcija"~ $\Gamma$ iz izreka~\ref{iz:funkcija_k} ni čista funkcija, 
saj obstajajo števila $n \geq 5$ za katere je $\Gamma(n)$ množica dveh števil.
Primeri takih (zaporednih) vrednosti $n$ 
so: 12, 117, 450, 4674, 48620, 505829, 1955714, 20347010, ...
\end{opomba}

Opremljeni s funkcijo $\Gamma$ prikažimo tabelo vrednosti $k$ za dane $n$, ki po 
trditvi~\ref{tr:domneva} tvorijo graf, ki doseže maksimalen DKZI. Ta tabela nadaljuje 
vrednosti tabele, ki so jo podali \textcite{saber2025}. \newline\newline
{\scriptsize
\begin{center}
\begin{tabular}{|c|c||c|c||c|c||c|c||c|c||c|c||c|c|}
\hline
$n$ & $k$ & $n$ & $k$ & $n$ & $k$ & $n$ & $k$ & $n$ & $k$ & $n$ & $k$ & $n$ & $k$\\ \hline
150 & 58 & 193 & 75 & 236 & 92 & 279 & 109 & 322 & 126 & 365 & 142 & 408 & 159\\ \hline
151 & 59 & 194 & 76 & 237 & 92 & 280 & 109 & 323 & 126 & 366 & 143 & 409 & 159\\ \hline
152 & 59 & 195 & 76 & 238 & 93 & 281 & 110 & 324 & 126 & 367 & 143 & 410 & 160\\ \hline
153 & 60 & 196 & 76 & 239 & 93 & 282 & 110 & 325 & 127 & 368 & 143 & 411 & 160\\ \hline
154 & 60 & 197 & 77 & 240 & 94 & 283 & 110 & 326 & 127 & 369 & 144 & 412 & 161\\ \hline
155 & 60 & 198 & 77 & 241 & 94 & 284 & 111 & 327 & 127 & 370 & 144 & 413 & 161\\ \hline
156 & 61 & 199 & 78 & 242 & 94 & 285 & 111 & 328 & 128 & 371 & 145 & 414 & 161\\ \hline
157 & 61 & 200 & 78 & 243 & 95 & 286 & 111 & 329 & 128 & 372 & 145 & 415 & 162\\ \hline
158 & 62 & 201 & 78 & 244 & 95 & 287 & 112 & 330 & 129 & 373 & 145 & 416 & 162\\ \hline
159 & 62 & 202 & 79 & 245 & 95 & 288 & 112 & 331 & 129 & 374 & 146 & 417 & 163\\ \hline
160 & 62 & 203 & 79 & 246 & 96 & 289 & 113 & 332 & 129 & 375 & 146 & 418 & 163\\ \hline
161 & 63 & 204 & 79 & 247 & 96 & 290 & 113 & 333 & 130 & 376 & 147 & 419 & 163\\ \hline
162 & 63 & 205 & 80 & 248 & 97 & 291 & 113 & 334 & 130 & 377 & 147 & 420 & 164\\ \hline
163 & 63 & 206 & 80 & 249 & 97 & 292 & 114 & 335 & 131 & 378 & 147 & 421 & 164\\ \hline
164 & 64 & 207 & 81 & 250 & 97 & 293 & 114 & 336 & 131 & 379 & 148 & 422 & 165\\ \hline
165 & 64 & 208 & 81 & 251 & 98 & 294 & 115 & 337 & 131 & 380 & 148 & 423 & 165\\ \hline
166 & 65 & 209 & 81 & 252 & 98 & 295 & 115 & 338 & 132 & 381 & 149 & 424 & 165\\ \hline
167 & 65 & 210 & 82 & 253 & 99 & 296 & 115 & 339 & 132 & 382 & 149 & 425 & 166\\ \hline
168 & 65 & 211 & 82 & 254 & 99 & 297 & 116 & 340 & 133 & 383 & 149 & 426 & 166\\ \hline
169 & 66 & 212 & 83 & 255 & 99 & 298 & 116 & 341 & 133 & 384 & 150 & 427 & 167\\ \hline
170 & 66 & 213 & 83 & 256 & 100 & 299 & 117 & 342 & 133 & 385 & 150 & 428 & 167\\ \hline
171 & 67 & 214 & 83 & 257 & 100 & 300 & 117 & 343 & 134 & 386 & 151 & 429 & 167\\ \hline
172 & 67 & 215 & 84 & 258 & 101 & 301 & 117 & 344 & 134 & 387 & 151 & 430 & 168\\ \hline
173 & 67 & 216 & 84 & 259 & 101 & 302 & 118 & 345 & 135 & 388 & 151 & 431 & 168\\ \hline
174 & 68 & 217 & 85 & 260 & 101 & 303 & 118 & 346 & 135 & 389 & 152 & 432 & 168\\ \hline
175 & 68 & 218 & 85 & 261 & 102 & 304 & 119 & 347 & 135 & 390 & 152 & 433 & 169\\ \hline
176 & 69 & 219 & 85 & 262 & 102 & 305 & 119 & 348 & 136 & 391 & 152 & 434 & 169\\ \hline
177 & 69 & 220 & 86 & 263 & 102 & 306 & 119 & 349 & 136 & 392 & 153 & 435 & 170\\ \hline
178 & 69 & 221 & 86 & 264 & 103 & 307 & 120 & 350 & 136 & 393 & 153 & 436 & 170\\ \hline
179 & 70 & 222 & 86 & 265 & 103 & 308 & 120 & 351 & 137 & 394 & 154 & 437 & 170\\ \hline
180 & 70 & 223 & 87 & 266 & 104 & 309 & 120 & 352 & 137 & 395 & 154 & 438 & 171\\ \hline
181 & 70 & 224 & 87 & 267 & 104 & 310 & 121 & 353 & 138 & 396 & 154 & 439 & 171\\ \hline
182 & 71 & 225 & 88 & 268 & 104 & 311 & 121 & 354 & 138 & 397 & 155 & 440 & 172\\ \hline
183 & 71 & 226 & 88 & 269 & 105 & 312 & 122 & 355 & 138 & 398 & 155 & 441 & 172\\ \hline
184 & 72 & 227 & 88 & 270 & 105 & 313 & 122 & 356 & 139 & 399 & 156 & 442 & 172\\ \hline
185 & 72 & 228 & 89 & 271 & 106 & 314 & 122 & 357 & 139 & 400 & 156 & 443 & 173\\ \hline
186 & 72 & 229 & 89 & 272 & 106 & 315 & 123 & 358 & 140 & 401 & 156 & 444 & 173\\ \hline
187 & 73 & 230 & 90 & 273 & 106 & 316 & 123 & 359 & 140 & 402 & 157 & 445 & 174\\ \hline
188 & 73 & 231 & 90 & 274 & 107 & 317 & 124 & 360 & 140 & 403 & 157 & 446 & 174\\ \hline
189 & 74 & 232 & 90 & 275 & 107 & 318 & 124 & 361 & 141 & 404 & 158 & 447 & 174\\ \hline
190 & 74 & 233 & 91 & 276 & 108 & 319 & 124 & 362 & 141 & 405 & 158 & 448 & 175\\ \hline
191 & 74 & 234 & 91 & 277 & 108 & 320 & 125 & 363 & 142 & 406 & 158 & 449 & 175\\ \hline
192 & 75 & 235 & 92 & 278 & 108 & 321 & 125 & 364 & 142 & 407 & 159 & 450 & \textbf{175,176}\\ \hline
\end{tabular}
\end{center}
}

\section{Grafi danega ciklomatičnega števila}

Problem maksimalne vrednosti drugega komplementarnega Zagrebškega indeksa pri danem številu 
vozlišč smo rešili. Optimalen graf, je dan s trditvama~\ref{tr:domneva} in~\ref{tr:glavna_k}.
Podobno je znana tudi optimalna vrednost s trditvama~\ref{tr:fnk} in~\ref{tr:glavna_k}.

V tem poglavju pa problem razširimo in se posvetimo grafom, ki za dano število vozlišč $n$ 
in dano število povezav $m$ dosežejo bodisi minimum bodisi maksimum drugega komplementarnega Zagrebškega 
indeksa. Ker se število povezav da enolično določiti s številom vozlišč in ciklomatičnem številom $\nu$,
si za lažje prepoznavanje vzorcev raje ogledujmo grafe reda $n$ ter ciklomatičnega števila $\nu$.

\begin{definicija}
  Naj bo $G$ graf reda $n$, $m$ število njegovih povezav ter $p$ število 
  povezanih komponent, ki sestavljajo graf $G$. Tedaj je ciklomatično 
  število $k$ grafa $G$ enako:
  \[\nu = m - n + p.\]
  Povezani grafi reda $n$ s ciklomatičnem številom $\nu$ imajo torej $m=n - 1 + \nu$ povezav.
\end{definicija}

\begin{trditev}
  Naj bo $G$ povezan graf reda $n$ ter s ciklomatičnem številom $\nu$, tedaj 
  ima $n \geq \lceil (3 + \sqrt(1 + 8\nu))/2 \rceil$ vozlišč. Namreč to sledi iz tega,
  da ima graf lahko največ toliko povezav kot bi jih imel, če bi bil poln.
\end{trditev}

\subsection{Problem minimuma}

Najprej se osredotočimo na grafe, ki pri danem $n$ in $\nu$ DKZI
minimizirajo. Manjše grafe poiščimo s pomočjo mešano celoštevilskega linearnega programiranja
(MILP),
večje pa poiščimo s preprosto heuristično metodo simulated annealing (SA).

MILP za problem mimimuma je sledeč: 
Naj bo za vsak urejen par $i < j$ spremenljivka $x_{ij}$ enaka 1 natanko tedaj 
ko sta vozlišči $i$ ter $j$ povezani. S $s_i$ modelirajmo kvadrat stopnje $i$-tega vozlišča.
S $z_{ij}$ modelirajmo absolutno vrednost razlike kvadratov stopenj. S spremenljivkami 
$f_{uv}$ pa modelirajmo povezanost grafa.

\begin{align}
&\min \sum_{i=1}^{n-1} \sum_{j=i+1}^{n} z_{ij} \\[1mm]
&\text{p. p.} \\[1mm]
&\sum_{i=1}^{n-1} \sum_{j=i+1}^{n} x_{ij} = m := n - 1 + \nu\\[1mm]
&\forall i \in \{1, \dots, n\}: \quad \sum_{t=1}^{n-1} y_{it} = 1 \\[1mm]
&\forall i \in \{1, \ldots, n\}: \quad \sum_{t=1}^{n-1}t y_{it} = \sum_{j < i}x_{ji} + \sum_{i < j}x_{ij} \\[1mm]
&\forall i \in \{1, \dots, n\}: \quad s_i = \sum_{t=1}^{n-1} t^2 \, y_{it} \\[1mm]
&\forall i,j \in \{1, \dots, n\}, \ i < j: \quad z_{ij} \ge s_i - s_j - (n-1)^2 (1 - x_{ij})\\[1mm]
&\forall i,j \in \{1, \dots, n\}, \ i < j: \quad z_{ij} \ge s_j - s_i - (n-1)^2 (1 - x_{ij})\\[1mm]
&\forall u,v \in \{1, \dots, n\}, \ u \neq v: \quad f_{uv} \le (n-1) x_{\min(u,v),\max(u,v)} \\[1mm]
&\forall v \in \{2, \dots, n\}: \quad \sum_{\substack{u=1 \\ u \neq v}}^{n} f_{uv} - \sum_{\substack{u=1 \\ u \neq v}}^{n} f_{vu} = 1 \\[1mm]
&\sum_{v=2}^{n} f_{1v} = n-1 \\[1mm]
&\forall i,j \in \{1, \dots, n\}, \ i < j: \quad x_{ij} \in \{0,1\} \\[1mm]
&\forall i,j \in \{1, \dots, n\}, \ i < j: \quad z_{ij} \ge 0 \\[1mm]
&\forall i \in \{1, \dots, n\}: \quad s_i \ge 0 \\[1mm]
&\forall i \in \{1, \dots, n\}, \forall t \in \{1, \dots, n-1\}: \quad y_{it} \in \{0,1\}
\end{align}

Najdeni grafi so sledečih oblik, pri čemer pa je vredno omeniti, da grafi za dan red $n$ ter 
ciklomatično število $\nu$ niso nujno unikatni. 

\begin{figure}[h]
    \centering
    \begin{subfigure}[b]{0.45\textwidth}
        \centering
        \includegraphics[width=\textwidth]{figures/conjecture_min_5_2.png}
        \caption{Graf reda 5 z ciklomatičnem številom 2. Primer, ki ga razreši prva točka hipoteze~\ref{hip:min}}
        \label{fig:5_2}
    \end{subfigure}
    \hfill
    \begin{subfigure}[b]{0.45\textwidth}
        \centering
        \includegraphics[width=\textwidth]{figures/conjecture_min_17_5.png}
        \caption{Graf reda 17 z ciklomatičnem številom 5. Primer, ki ga razreši tretja točka hipoteze~\ref{hip:min}}
        \label{fig:17_5}
    \end{subfigure}
    \caption{Grafa, ki glede na svoje število vozlišč ter ciklomatično število dosežeta minimum DKZI vrednosti.}
    \label{fig:hip_min_zgled}
\end{figure}

\begin{hipoteza}\label{hip:min}
  Z $G_{n, \nu}$ označimo graf reda $n$, ki za dano ciklomatično število $\nu$ doseže
  minimalno vrednost DKZI, tedaj velja:
  \begin{enumerate}
    \item $\cM(G_{4, 2}) = 20 \land \cM(G_{5, 2}) = 18 \land \quad \forall n\geq 6:\ \cM(G_{n, 2}) = 16$
    \item $\forall \nu\geq 3:\ \cM(G_{2\nu - 1, \nu}) = 10$.
    \item $\forall \nu\geq 3, \forall n\geq 2\nu:\ \cM(G_{n, \nu}) = 8$.
  \end{enumerate}
\end{hipoteza}

\subsection{Problem maksimuma}

\begin{opomba}
  Ciklomatično število grafa $K_k + \KC_{n-k}$ je enako 
  \[\nu = k (n-k) + k (k-1) / 2 - n + 1.\] To preprosto sledi iz števila 
  njegovih povezav. 
\end{opomba}

\begin{hipoteza}\label{hip:max}
  Z $G_{n, \nu}$ označimo graf reda $n$, ki za dano ciklomatično število $\nu$ doseže maksimalno 
  vrednost DKZI, tedaj velja:
  \begin{enumerate}
    \item Naj bo $Z_n$ zvezda, $\KC_n$ prazen graf ter $K_1$ enovozliščen graf. Za 
    poljuben $\nu$ in za vsak $n \geq \nu + 2$ velja 
    $G_{n, \nu} = (Z_{\nu} \cup \KC_{n - \nu - 2}) + K_1$ in 
    $\cM(G_{n, \nu}) = n(n-1)(n-2) + \nu(\nu^2 + \nu - 8)$.
    \item Naj bo $\lceil (3 + \sqrt(1 + 8\nu))/2 \rceil \leq n < \nu + 2$. 
    Naj bo $(a, b)$ prvi celoštevilski par, ki zadošča enačbi $b = (1+a)(4+a-2n) / 2 + \nu$,
    in pogojema $0 \leq a \leq n - 3$ ter $1 \leq b \leq n - 1 - a$. Tedaj velja:
    \begin{align*}
      \cM(G_{n, \nu}) &= ((n-1)^2 - (2+a+b)^2) \ (2 + a) + \\
                      &+ ((2+a+b)^2 - (3+a)^2) \ b \\
                      &+ ((n-1)^2 - (3+a)^2) \ 2 b \\
                      &+ ((n-1)^2 - (3+a)) \ a b \\ 
                      &+ ((n-1)^2 - (2+a)^2) \ 2 (n-3-a-b) \\
                      &+ ((n-1)^2 - (2+a)^2) \ a (n-3-a-b)
    \end{align*}
  \end{enumerate}
\end{hipoteza}

\begin{figure}[h]
    \centering
    \begin{subfigure}[b]{0.45\textwidth}
        \centering
        \includegraphics[width=\textwidth]{figures/conjecture_max_14_9.png}
        \caption{Graf reda 14 z ciklomatičnem številom 9. Primer, ki ga razreši prva točka hipoteze~\ref{hip:max}}
        \label{fig:14_9}
    \end{subfigure}
    \hfill
    \begin{subfigure}[b]{0.45\textwidth}
        \centering
        \includegraphics[width=\textwidth]{figures/conjecture_max_8_9.png}
        \caption{Graf reda 8 z ciklomatičnem številom 9. Primer, ki ga razreši druga točka hipoteze~\ref{hip:max}}
        \label{fig:8_9}
    \end{subfigure}
    \caption{Grafa, ki glede na svoje število vozlišč ter ciklomatično število dosežeta maksimum DKZI vrednosti.}
    \label{fig:hip_max_zgled}
\end{figure}

\begin{opomba}
  Hipoteza~\ref{hip:max} v celoti razreši problem maksimalnega DKZI pri grafih 
  z danim redom $n$ ter ciklomatičnem številom $\nu$.
\end{opomba}
Ob predpostavki, da hipoteza velja, prikažimo tabelo maksimalnih DKZI vrednosti za dana 
parametra $n$ ter $\nu$. Na njej pa odebeljeno označimo tiste vrednosti DKZI, ki 
nam jih narekuje izrek~\ref{iz:funkcija_k}. Vidimo, da se naša hipoteza z njimi ujema
povsod koder se lahko. Prav tako se hipoteza ujema z vsemi rezultati, ki jih 
prineseta metodi (MILP) in (SA).
{\scriptsize
\begin{center}
\begin{tabular}{|c|c|c||c|c|c||c|c|c||c|c|c|}
\hline
$n$ & $\nu$ & $\cM(G_{n, \nu})$ & $n$ & $\nu$ & $\cM(G_{n, \nu})$ & $n$ & $\nu$ & $\cM(G_{n, \nu})$ & $n$ & $\nu$ & $\cM(G_{n, \nu})$\\ \hline
3 & 1 & 0 & 6 & 5 & 148 & 14 & 9 & 2922 & 14 & 14 & 3930\\ \hline
4 & 1 & 18 & 7 & 5 & 320 & 15 & 9 & 3468 & 15 & 14 & 4972\\ \hline
5 & 1 & 54 & 8 & 5 & 446 & 16 & 9 & 4098 & 16 & 14 & 6188\\ \hline
6 & 1 & 114 & 9 & 5 & 614 & 17 & 9 & 4818 & 17 & 14 & 6908\\ \hline
7 & 1 & 204 & 10 & 5 & 830 & 18 & 9 & 5634 & 18 & 14 & 7724\\ \hline
8 & 1 & 330 & 11 & 5 & 1100 & 6 & 10 & 0 & 7 & 15 & 0\\ \hline
9 & 1 & 498 & 12 & 5 & 1430 & 7 & 10 & 282 & 8 & 15 & 528\\ \hline
10 & 1 & 714 & 13 & 5 & 1826 & 8 & 10 & 544 & 9 & 15 & 918\\ \hline
11 & 1 & 984 & 14 & 5 & 2294 & 9 & 10 & 816 & 10 & 15 & 1512\\ \hline
12 & 1 & 1314 & 15 & 5 & 2840 & 10 & 10 & 1202 & 11 & 15 & 1878\\ \hline
13 & 1 & 1710 & 16 & 5 & 3470 & 11 & 10 & 1708 & 12 & 15 & 2400\\ \hline
14 & 1 & 2178 & 17 & 5 & 4190 & 12 & 10 & 2340 & 13 & 15 & 3084\\ \hline
15 & 1 & 2724 & 18 & 5 & 5006 & 13 & 10 & 2736 & 14 & 15 & 3936\\ \hline
16 & 1 & 3354 & 5 & 6 & 0 & 14 & 10 & 3204 & 15 & 15 & 4962\\ \hline
17 & 1 & 4074 & 6 & 6 & 138 & 15 & 10 & 3750 & 16 & 15 & 6168\\ \hline
18 & 1 & 4890 & 7 & 6 & 300 & 16 & 10 & 4380 & 17 & 15 & 7560\\ \hline
4 & 2 & 20 & 8 & 6 & 540 & 17 & 10 & 5100 & 18 & 15 & 8376\\ \hline
5 & 2 & 56 & 9 & 6 & 708 & 18 & 10 & 5916 & 8 & 16 & 456\\ \hline
6 & 2 & 116 & 10 & 6 & 924 & 7 & 11 & 252 & 9 & 16 & 906\\ \hline
7 & 2 & 206 & 11 & 6 & 1194 & 8 & 11 & \textbf{600} & 10 & 16 & 1470\\ \hline
8 & 2 & 332 & 12 & 6 & 1524 & 9 & 11 & 844 & 11 & 16 & 2008\\ \hline
9 & 2 & 500 & 13 & 6 & 1920 & 10 & 11 & 1208 & 12 & 16 & 2490\\ \hline
10 & 2 & 716 & 14 & 6 & 2388 & 11 & 11 & 1698 & 13 & 16 & 3140\\ \hline
11 & 2 & 986 & 15 & 6 & 2934 & 12 & 11 & 2320 & 14 & 16 & 3964\\ \hline
12 & 2 & 1316 & 16 & 6 & 3564 & 13 & 11 & 3080 & 15 & 16 & 4968\\ \hline
13 & 2 & 1712 & 17 & 6 & 4284 & 14 & 11 & 3548 & 16 & 16 & 6158\\ \hline
14 & 2 & 2180 & 18 & 6 & 5100 & 15 & 11 & 4094 & 17 & 16 & 7540\\ \hline
15 & 2 & 2726 & 6 & 7 & 144 & 16 & 11 & 4724 & 18 & 16 & 9120\\ \hline
16 & 2 & 3356 & 7 & 7 & 290 & 17 & 11 & 5444 & 8 & 17 & 398\\ \hline
17 & 2 & 4076 & 8 & 7 & 520 & 18 & 11 & 6260 & 9 & 17 & 918\\ \hline
18 & 2 & 4892 & 9 & 7 & 840 & 7 & 12 & 240 & 10 & 17 & 1440\\ \hline
4 & 3 & 0 & 10 & 7 & 1056 & 8 & 12 & 558 & 11 & 17 & 2184\\ \hline
5 & 3 & \textbf{72} & 11 & 7 & 1326 & 9 & 12 & 900 & 12 & 17 & 2620\\ \hline
6 & 3 & 132 & 12 & 7 & 1656 & 10 & 12 & 1236 & 13 & 17 & 3230\\ \hline
7 & 3 & 222 & 13 & 7 & 2052 & 11 & 12 & 1704 & 14 & 17 & 4020\\ \hline
8 & 3 & 348 & 14 & 7 & 2520 & 12 & 12 & 2310 & 15 & 17 & 4996\\ \hline
9 & 3 & 516 & 15 & 7 & 3066 & 13 & 12 & 3060 & 16 & 17 & 6164\\ \hline
10 & 3 & 732 & 16 & 7 & 3696 & 14 & 12 & 3960 & 17 & 17 & 7530\\ \hline
11 & 3 & 1002 & 17 & 7 & 4416 & 15 & 12 & 4506 & 18 & 17 & 9100\\ \hline
12 & 3 & 1332 & 18 & 7 & 5232 & 16 & 12 & 5136 & 8 & 18 & 360\\ \hline
13 & 3 & 1728 & 6 & 8 & 102 & 17 & 12 & 5856 & 9 & 18 & 960\\ \hline
14 & 3 & 2196 & 7 & 8 & 296 & 18 & 12 & 6672 & 10 & 18 & 1428\\ \hline
15 & 3 & 2742 & 8 & 8 & 510 & 7 & 13 & 168 & 11 & 18 & 2142\\ \hline
16 & 3 & 3372 & 9 & 8 & 820 & 8 & 13 & 528 & 12 & 18 & 2796\\ \hline
17 & 3 & 4092 & 10 & 8 & 1232 & 9 & 13 & \textbf{990} & 13 & 18 & 3360\\ \hline
18 & 3 & 4908 & 11 & 8 & 1502 & 10 & 13 & 1292 & 14 & 18 & 4110\\ \hline
5 & 4 & 52 & 12 & 8 & 1832 & 11 & 13 & 1732 & 15 & 18 & 5052\\ \hline
6 & 4 & \textbf{168} & 13 & 8 & 2228 & 12 & 13 & 2316 & 16 & 18 & 6192\\ \hline
7 & 4 & 258 & 14 & 8 & 2696 & 13 & 13 & 3050 & 17 & 18 & 7536\\ \hline
8 & 4 & 384 & 15 & 8 & 3242 & 14 & 13 & 3940 & 18 & 18 & 9090\\ \hline
9 & 4 & 552 & 16 & 8 & 3872 & 15 & 13 & 4992 & 8 & 19 & 250\\ \hline
10 & 4 & 768 & 17 & 8 & 4592 & 16 & 13 & 5622 & 9 & 19 & 888\\ \hline
11 & 4 & 1038 & 18 & 8 & 5408 & 17 & 13 & 6342 & 10 & 19 & 1440\\ \hline
12 & 4 & 1368 & 6 & 9 & 72 & 18 & 13 & 7158 & 11 & 19 & 2112\\ \hline
13 & 4 & 1764 & 7 & 9 & \textbf{324} & 7 & 14 & 110 & 12 & 19 & 3024\\ \hline
14 & 4 & 2232 & 8 & 9 & 516 & 8 & 14 & 516 & 13 & 19 & 3536\\ \hline
15 & 4 & 2778 & 9 & 9 & 810 & 9 & 14 & 948 & 14 & 19 & 4240\\ \hline
16 & 4 & 3408 & 10 & 9 & 1212 & 10 & 14 & 1382 & 15 & 19 & 5142\\ \hline
17 & 4 & 4128 & 11 & 9 & 1728 & 11 & 14 & 1788 & 16 & 19 & 6248\\ \hline
18 & 4 & 4944 & 12 & 9 & 2058 & 12 & 14 & 2344 & 17 & 19 & 7564\\ \hline
5 & 5 & 42 & 13 & 9 & 2454 & 13 & 14 & 3056 & 18 & 19 & 9096\\ \hline
\end{tabular}
\end{center}
}


\section{Preverjanje manjših grafov}
V tem razdelku empirično preverimo doseganje minimalnih in maksimalnih vrednosti drugega komplementarnega 
Zagrebškega indeksa $\cM(G)$ za vse povezane grafe majhnega reda $n$.
Postopek je sledeč:
\begin{enumerate}
\item Sistematično generiramo vse neizomorfne povezane grafe za izbrana števila vozlišč ($n = 3, 4, 5, 6, 7, 8$).
\item Za vsak graf izračunamo $\cM(G)$ po definiciji.
\item Zabeležimo grafe z minimalno in maksimalno vrednostjo indeksa.
\item Preverimo, ali ima graf z maksimalnim indeksom obliko $K_{k} + \overline{K}_{n-k}$, kot napoveduje izrek.
\end{enumerate}

V spodnji tabeli je za vsako število vozlišč $n$ prikazana minimalna in maksimalna dosežena vrednosti 
komplementarnega drugega zagrebškega indeksa $\cM(G)$ ter pripadajoč parameter $k$.

\begin{tabular}{|c|c|c|c|c|c|}
\hline
n & Min $cM_2$ & Je join & $k$ & Max $cM_2$ & $k$ \\
\hline
3 & 0 & Da & 2 & 6 & 1 \\
4 & 0 & - & - & 24 & 1 \\
5 & 0 & - & - & 72 & 2 \\
6 & 0 & - & - & 168 & 2 \\
7 & 0 & - & - & 324 & 3 \\
8 & 0 & - & - & 600 & 3 \\
\hline
\end{tabular}

Pri minimalnih vrednostih indeksa se pogosto pojavljajo grafi, kjer je večina vozlišč povezanih oziroma 
je skoraj celoten graf popoln, medtem ko je preostali del izoliran. Zaradi tega so razlike med kvadrati 
stopenj sosednjih vozlišč zelo majhne ali celo ničelne. Iz tabele je razvidno, da za 
vse obravnavane vrednosti $n \leq 8$ minimalno vrednost indeksa $\cM(G)$ dosežejo grafi, ki so 
regularni ali ciklični, saj imajo vozlišča enako stopnjo, kar zmanjša razliko kvadratov stopenj na 
povezavah na nič. Posledično je minimalna vrednost indeksa enaka nič.

Pri maksimalnih vrednostih indeksa pa prevladujejo grafi, ki so izomorfni grafu oblike $K_k + \overline{K}_{n-k}$, 
torej sestavljeni iz popolnega grafa z $k$ vozlišči in nepovezanih vozlišč z ostankom $n-k$, med katerima 
so vse možne povezave. Ti grafi zaradi močne razlike v stopnjah vozlišč med komponentama dosegajo največje
vrednosti indeksa. Iz tabele je razvidno da so bile maksimalne vrednosti vedno dosežene pri takšnih strukturah, 
kar je skladno z trditvijo~\ref{tr:domneva}.

Na slikah so prikazani minimalni in maksimalni grafi za $n = 3, 4, 5$.  
% SLIKE ZA N = 3,4,5

\section{Metaheuristične metode}

Metaheuristika je množica algoritemskih konceptov in strategij visoke ravni, ki se uporabljajo
za definiranje heuristične metode, primerne za širok nabor optimizacijskih problemov. Iščejo dovolj
dobre rešitve kompleksnih problemov, kjer je podrobno iskanje najboljše rešitve nemogoče ali pa traja predolgo.

Za večje grafe ($n \geq 20$), kjer točne metode (MILP) postanejo računsko prezahtevne, sv uporabili metahevristiko 
\textbf{Simulated Annealing}. To je stohastična optimizacijska metoda, ki začne z naključno rešitvijo 
in jo iterativno izboljšuje z zamenjavo povezav v grafu. Na začetku (pri visoki temperaturi) algoritem 
sprejema tudi slabše rešitve, kar omogoča izogibanje lokalnim optimumom. S časom temperatura pada in 
algoritem konvergira k dobri rešitvi.






\section{Vzorci za minimalne in maksimalne grafe}

Pri minimalnih grafih za ciklomatično število $k$ so minimalni grafi strukturirani kot cikel $C_n$ z $k$ dodatnimi 
povezavami. Te dodatne povezave so razporejene enakomerno po ciklu, kar zagotavlja majhne razlike med stopnjami 
sosednjih vozlišč. Pri večjih $n$ ta vzorec postane stabilen, saj si grafi postajajo strukturno podobni.

Za maksimalne grafe pri majhnih $n$ maksimalne vrednosti dosegajo grafi z zvezdasto strukturo, kjer ima eno centralno 
vozlišče visoko stopnjo, ostala pa nizko. Ko $n$ narašča, se struktura razvije v join graf oblike $K_j + \overline{K}_{n-j}$. 
Tu je $K_j$ poln graf z $j$ vozlišči, $\overline{K}_{n-j}$ pa množica izoliranih vozlišč, povezanih le na $K_j$. 
Ta struktura ustvari velika razlika med stopnjami: vozlišča v $K_j$ imajo stopnjo blizu $n$, zunanja pa okoli $j$. 
Pri večjih $k$ se začnejo dodajati povezave tudi med zunanjimi vozlišči, vendar osnovna zvezdasta oblika ostane prepoznavna.



\end{document}

