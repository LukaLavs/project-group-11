\documentclass[a4paper,12pt]{report}

\usepackage[utf8]{inputenc}
\usepackage[T1]{fontenc}
\usepackage{amsmath, amssymb}
\usepackage{graphicx}
\usepackage{hyperref}

\begin{document}


\begin{center}
    \large \textbf{Komplementarni drugi Zagrebški indeks}\\
    \large Kratek opis problema in načrt dela\\[1em]
    Luka Lavš, Tinka Napret-Kaučič\\
    November, 2025
\end{center}



V projektni nalogi z naslovom \textbf{Komplementarni drugi Zagrebški indeks} bova 
najprej opredelila indeks, nato bova previla konjunkturo. Ob predpostavki, da 
je konjunktura pravilna, določila $\mathit{k}$ kot funkcijo odvisno od $\mathit{n}$.
Prav tako bova poiskala kateri grafi imajo najmanjši in najveljo vrednost $cM_2(G)$
nad grafi z $\mathit{n}$ vozlišči s ciklomatičnim številom $\mathit{k}$.


Problem je zanimiv tako z matematičnega kot z računalniškega vidika, saj združuje teorijo 
grafov, kombinatoriko in optimizacijo. Dodatno se uporablja tudi v kemijskih in omrežnih 
analizah, kjer opisuje heterogenost vozlišč.

\section*{Opredelitev indeksa}

Indeks komplementarne druge zagrebške lestvice (CSZ indeks) za graf $G$ je definiran kot
\[
cM_2(G) = \sum_{uv \in E(G)} \left| d_G(u)^2 - d_G(v)^2 \right|,
\]
kjer $d_G(u)$ označuje stopnjo vozlišča $u$ v grafu $G$, $E(G)$ pa predstavlja množico povezav grafa $G$.
Opozorimo, da znak $+$ v nadaljevanju pomeni \textit{join} dveh grafov.

CSZ indeks sodi med \textit{Zagrebške indekse}, ki se uporabljajo za opisovanju strukturnih lastnosti grafov.
Indeksi so nastali na področju kemijske teorije grafov, kjer predstavljajo mero razvejanosti molekul,
v sodobni teoriji grafov pa se uporabljajo za opis topoloških značilnosti in kompleksnosti omrežij.

Za razliko od klasičnega drugega Zagreb indeksa, ki se definira kot vsota produktov stopenj sosednjih vozlišč,
CSZ indeks meri razliko med kvadrati stopenj vozlišč, povezanih z robom. S tem indeks bolje zazna neuravnoteženost 
v lokalni strukturi grafa, torej kako različne so stopnje sosednjih vozlišč. Pri višjih vrednosti indeksa $cM_2(G)$ 
so razlike v stopnjah med sosednjimi vozlišči večje, medtem ko manjše vrednosti predstavljajo bolj homogeno strukturo.


\section*{Načrt dela}

Ideja prve faze projekta je analiza in preizkušanje lastnosti grafov glede na njihovo vrednost
\textit{Complementary Second Zagreb Indexa (CSZ indeksa)}. Najprej bova implementirala funkcijo, ki za dani graf $G$
izračuna vrednost $cM_2(G)$, pri čemer uporabimo formulo
\[
cM_2(G) = \sum_{uv \in E(G)} |d_G(u)^2 - d_G(v)^2|,
\]
kjer $d_G(u)$ označuje stopnjo vozlišča $u$ v grafu. Funkcija bo preverjala vse povezave v grafu in za vsako sosednjo
povezavo izračunala prispevek k skupni vrednosti indeksa. Na podlagi dobljenih vrednosti bova v prvi fazi sistematično 
preverili grafe manjših redov, da bi ugotovila, kateri grafi imajo največjo oziroma najmanjšo vrednost $cM_2$. Poseben 
poudarek bo na preverjanju konjekture, da graf, ki maksimizira $cM_2$, ustreza strukturi $K_k + K_{n-k}$ za ustrezen 
$k < \lceil n/2 \rceil$.

Nato bova za iskanje ekstremnih grafov uporabila \textit{metahevristične metode}. Gre za skupino stohastičnih optimizacijskih 
pristopov, ki z uporabo naključnosti iščejo čim boljše rešitve v primerih, kjer popolno iskanje postane računsko neizvedljivo. 
Pri večjih grafih boma uporabila tri metode: genetski algoritem, simulirano ohlajanje in pohlepno iskanje. Namen teh metod ni nujno 
najti popolno rešitev, temveč dovolj dobro rešitev v razumnem času.



\end{document}
